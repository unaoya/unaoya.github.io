\documentclass[uplatex]{jsarticle}
\usepackage{longtable}
\title{数学について話す会プログラム}
\begin{document}
\date{}
\maketitle
\begin{longtable}{ll}
10:30-10:35 & 諸注意\\
10:35-10:50 & 順序数について\\
& N.Y@N\_Y\_Big\_Apple\\
10:50-11:05 & 数学の定理を物理的に解釈\\
& 三島太郎@hdfghgftrr\\
11:05-11:20 & 機械学習の話\\
& あり@ta\_to\_co\\
11:20-11:30 & 休憩\\
11:30-11:45 & [ Hochshild | Magnitude | Persistent ] Homology\\
& s.t.@simizut22\\
11:45-12:00 & 計算論とライスの定理について\\
& λx.x@lambda\_x\_x\\
12:00-12:15 & 一昨日 arXiv にあがった論文について\\
& せきゅーん@integers\_blog\\
12:15-13:15 & 昼休み\\
13:15-13:30 & たのしいけんろん!\\
& ろりじょ@R\_O\_R\_I\_J\_O\\
13:30-13:45 & 合同数とTunnellの定理\\
& tsujimotter@tsujimotter\\
13:45-14:00 & グラフ理論の基礎と確率論的手法について\\
& 藤井\\
14:00-14:15 & 休憩\\
14:15-14:30 & 虚数乗法勉強中\\
& 松森至宏@yoshi\_matsumori\\
14:30-14:45 & 実二次体の類数公式について\\
& 中澤俊彦\\
14:45-15:00 & Chowla-Selberg公式について\\
& 梅崎直也@unaoya\\
15:00-15:15 & 休憩\\
\newpage
15:15-15:30 & 周期とKontsevich-Zagier予想について\\
& @m\_river\_@m\_river\_\\
15:30-15:45 & Asymptotic City\\
& たけのこ赤軍@691\_7758337633\\
15:45-16:00 & knotとstring、qとQ\\
& nw@math\_phys\\
16:00-16:15 & 休憩\\
16:15-16:30 & 順序数について\\
& N.Y@N\_Y\_Big\_Apple\\
16:30-17:45 & 多元環と道代数\\
& サクラ@1997\_takahashi\\
16:45-17:00 & 8次元球充填について\\
& せきゅーん@integers\_blog\\
17:00- & 懇親会
\end{longtable}
\end{document}